\documentclass{article}
\usepackage{graphicx} % Required for inserting images
\usepackage{hyperref}
\usepackage{caption} % For custom captions
\usepackage{amsmath}
\usepackage{multicol}
\usepackage{subcaption}


\title{Fundamentals of Robotics: Assignment 1}
\author{Ekaterina Mozhegova}
\date{April 12, 2024}

\begin{document}

\maketitle

\section{Link}

\href{https://colab.research.google.com/drive/170z2YVaeu9kHEuC4bXIWQqJ82eWDJoS2?usp=sharing}{Colab}

\section{Task 1}

\subsection{Task Description}

\begin{enumerate}
    \item Select a robot's elbow model. 
    \item Couple your elbow model with a spherical wrist that satisfies Euler angles arrangement (xyz, xzy,
xyx, xzx, yxz, yzx, yxy, yzy, zxy, zyx, zyz, zxz) (Make sure it is a spherical wrist)
    \item Solve Forward kinematics problem (using DH-Parameter is optional). You can use 6
transformations between two frames.
\end{enumerate}


\subsection{Solution}

\subsubsection{Elbow}

The chosen elbow model - \textbf{Antropomorphic}.
\begin{center}
    \includegraphics*[width=0.65\textwidth]{images/new_elbow.png}
\end{center}


Let's describe the transformation matrices for transitions between each coordinate frame of  the elbow:

\[{}^0T_1 = R_z(\theta_1^*) T_z(l_1) R_z(\frac{\pi}{2}) R_x(\frac{\pi}{2})\]
\[{}^1T_2 = R_z(\theta_2^*) T_x(l_2) \]
\[{}^2T_3 = R_z(\theta_3^*) T_x(l_3) \]


Thus, the transformation matrix for the elbow is as follows:
\[{}^0T_3 = {}^0T_1 {}^1T_2 {}^2T_3 \]



\subsubsection{Spherical wrist}

\begin{center}
    \includegraphics*[width=0.85\textwidth]{images/new_spherical_wrist.png}
\end{center}

The wrist is spherical since all revolute axes intersect in a single point. 

Let's describe the transformation matrices for transitions between each coordinate frame of  the wrist:
\[{}^3T_4 = R_z(\theta_4^*) T_z(l_4) R_x(-\dfrac{\pi}{2}) R_z(-\pi)\]
\[{}^4T_5 = R_z(\theta_5^*) T_y(l_5) R_x(-\dfrac{\pi}{2})\]
\[{}^5T_6 = R_z(\theta_6^*)*T_z(l_6)\]

Thus, the transformation matrix for the wrist is as follows:
\[{}^3T_6 = {}^3T_4 {}^4T_5 {}^5T_6 \]

This spherical wrist is associated with the $ZYZ$ Euler angles arrangement.


\subsubsection{Manipulator: antropomorphic elbow coupled with a spherical wrist}

The assembly of the manipulator with the spherical wrist coupled to the elbow is as follows.

\begin{center}
    \includegraphics*[width=0.85\textwidth]{images/new_manipulator.png}
\end{center}

This configuration is described by B. Siciliano in \emph{Robotics: Modelling, Planning, and Control}.

\subsubsection{Forward kinematics}


There are two approaches for solving forward kinematics problem: 

\begin{enumerate}
    \item 
    For this manipulator, consisting of an anthropomorphic arm with a spherical wrist,
    the direct kinematics problem cannot be obtained by pure multiplication of transformation matrices ${}^0T_3$ and ${}^3T_6$ since 
    the elbow needs to coincide with the spherical wrist. 
    We need to transform from the final reference frame of the elbow to the base reference frame of the wrist.
    So, one more transformation matrix has to be introduced, this matrix will describe transition between those two reference frames.
    \item We multiply all the transformation matrices - transitions between the reference frames.
\end{enumerate}


\underline{\textbf{Method 1}}\\

Let's solve the Forward Kinematics problem via the \textbf{method 1}.

We already defined the transformation matrices for the elbow and for the wrist - ${}^0T_3$, ${}^3T_6$. 

Now we need to introduce matrix $T^*$ defining the transition from the elbow to the wrist.

\[T^* = R_y(\dfrac{\pi}{2}) R_z(\dfrac{\pi}{2})\]


Thus, the transformation matrix for the end-effector is as follows:
\[{}^0T_6 = {}^0T_3 \cdot T^* \cdot {}^3T_6\]




\underline{\textbf{Method 2}}\\

Now let's solve the Forward Kinematics problem via the \textbf{method 2}.

The consecutive multiplication of the transformation matrices between the reference frames is as follows:

\[{}^0T_1 = R_z(\theta_1^*) T_z(l_1) R_z(\frac{\pi}{2}) R_x(\frac{\pi}{2})\]
\[{}^1T_2 = R_z(\theta_2^*) T_x(l_2) \]
\[{}^2T_3 = R_z(\theta_3^*) T_x(l_3) R_y(\dfrac{\pi}{2}) R_z(\dfrac{\pi}{2})\]
\[{}^3T_4 = R_z(\theta_4^*) T_z(l_4) R_x(-\dfrac{\pi}{2}) R_z(-\pi)\]
\[{}^4T_5 = R_z(\theta_5^*) T_y(l_5) R_x(-\dfrac{\pi}{2})\]
\[{}^5T_6 = R_z(\theta_6^*) T_z(l_6)\]


Thus, the transformation matrix for the end-effector is as follows:
\[{}^0T_6 = {}^0T_1 {}^1T_2 {}^2T_3 {}^3T_4 {}^4T_5 {}^5T_6\]



Solving symbolically, we come up with ${}^0T_6$. For this, please, check Colab.\\





\newpage

\subsubsection{Different manipulator's configrations}

Please, refer to the code in colab link to check solutions for different manipulator's configrations.

\end{document}