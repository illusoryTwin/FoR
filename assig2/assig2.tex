\documentclass{article}
\usepackage{graphicx} % Required for inserting images
\usepackage{hyperref}
\usepackage{caption} % For custom captions
\usepackage{amsmath}
\usepackage{multicol}
\usepackage{subcaption}
\usepackage{amsmath}

\title{Fundamentals of Robotics: Assignment 2}
\author{Ekaterina Mozhegova}
\date{April 20, 2024}

\begin{document}

\maketitle

\section{Link}

\href{https://colab.research.google.com/drive/1hU13PXIqbAtPv2cby5qNB10PobzW39JC?usp=sharing}{Colab}

\section{Configuration of the chosen robot}

Manipulator with antropomorphic elbow and a spherical wrist. 

\begin{center}
    \includegraphics*[width=0.85\textwidth]{images/new_manipulator.png}
\end{center}

Forward kinematics:
\[{}^0T_1 = R_z(\theta_1^*) T_z(l_1) R_z(\frac{\pi}{2}) R_x(\frac{\pi}{2})\]
\[{}^1T_2 = R_z(\theta_2^*) T_x(l_2) \]
\[{}^2T_3 = R_z(\theta_3^*) T_x(l_3) \]
\[{}^3T_4 = R_z(\theta_4^*) T_z(l_4) R_x(-\dfrac{\pi}{2}) R_z(-\pi)\]
\[{}^4T_5 = R_z(\theta_5^*) T_y(l_5) R_x(-\dfrac{\pi}{2})\]
\[{}^5T_6 = R_z(\theta_6^*) T_z(l_6) \]

\[{}^0T_6 = {}^0T_1 {}^1T_2 {}^2T_3 {}^3T_4 {}^4T_5 {}^5T_6 \]

% Let's define its geometry:
% \[l_1 = 1, l_2 = 0.5, l_3 = 0.4, l_4 =0.4, l_5=0.5, l_6 = 0.4\]

% And let's define the end-effector position:
% \[x_6=1.5, y_6=0.5, z_6 = 1.5\]


\section{Task 1-3}
\textbf{Task 1. Derive inverse kinematics for your robot model.\\}
\textbf{Task 2. Solve inverse kinematics for multiple positions.\\}
\textbf{Task 3. Track the number of solutions along the way and choose the correct one and closest to
the previous (current) configuration.}\\

\subsection{$q_1$}

\underline{Given:} End-effector position and orientation ${}^0T_6$.\\

First, let's define the configuration of the wrist center by solving forward kinematics. 
The end-effector position corresponds to the position of the wrist center; 
the 6th joint defines the end-effector orientation. 

\begin{align*}
    T_{01} &= R_z(\theta_1) \cdot T_z(l_1) \cdot R_z(\frac{\pi}{2}) \cdot R_x(\frac{\pi}{2}) \\
    T_{12} &= R_z(\theta_2) \cdot T_x(l_2) \\
    T_{23} &= R_z(\theta_3) \cdot T_x(l_3 + l_4) \cdot R_y(\frac{\pi}{2}) \cdot R_z(\frac{\pi}{2}) \\
    T_{\text{wrist\_center}} &= T_0 \cdot T_{01} \cdot T_{12} \cdot T_{23}
\end{align*}

Let's solve the forward kinematics problem for the wrist center symbolically.
Notice the translation column of the transformation matrix.  
From it we can retrieve the value of $q_1 = \theta_1$.


\[
\begin{smallmatrix}
    x \\ y \\ z
\end{smallmatrix}
=
\begin{smallmatrix}
    0.8 \sin(\theta_1) \sin(\theta_2) \sin(\theta_3) - 0.8 \sin(\theta_1) \cos(\theta_2) \cos(\theta_3) - 1.0 \sin(\theta_1) \cos(\theta_2) \\
    -0.8 \sin(\theta_2) \sin(\theta_3) \cos(\theta_1) + 0.8 \cos(\theta_1) \cos(\theta_2) \cos(\theta_3) + 1.0 \cos(\theta_1) \cos(\theta_2) \\
    0.8 \sin(\theta_2) \cos(\theta_3) + 1.0 \sin(\theta_2) + 0.8 \sin(\theta_3) \cos(\theta_2) + 0.3
\end{smallmatrix}
\]

We can see that:
\[\frac{x}{y} = -\frac{\sin(\theta_1)}{\cos(\theta_1)}\]

\[\tan(\theta_1) = -\dfrac{x}{y}\]
\[\theta_1 = -\arctan(\dfrac{x}{y})\]

\subsection{$q_3$}

We can find the value of $q_3 = \theta_3$ by the cosine theorem. 

\[q3 = \arccos\left(\frac{{(z-l1)^2 + (x^2+y^2) - l2^2 - (l3+l4)^2}}{{2 \cdot l2 \cdot (l3+l4)}}\right)\]

\subsection{$q_2$}

\[r = \sqrt{x^2 + y^2}\]

\[s = z - l1\]

\[q_2 = \arcsin\left(\frac{{(l2 + (l3 + l4)\cos(q_3)) \cdot s - (l3 + l4)\sin(q_3) \cdot r}}{{r^2 + s^2}}\right)\]


\subsection{$q_4$}

We know that 
\[{}^0R_6 = {}^0R_3 \cdot {}^3R_6\]
\[{}^3R_6 = ({}^0R_3^T) {}^0R_6\]
We have already calculated the angles $q_1$, $q_2$, and $q_3$, so we can calculate ${}^0R_3^T$. Additionally, we know ${}^0R_6$ since the end-effector position is provided to us. 
Thus, with ${}^3R_6$, we can determine all remaining angles.


Let's solve ${}^3R_6$ symbolically, 

\[
\begin{smallmatrix}
\sin(\theta_4)\sin(\theta_6) - \cos(\theta_4)\cos(\theta_5)\cos(\theta_6) & \sin(\theta_4)\cos(\theta_6) + \sin(\theta_6)\cos(\theta_4)\cos(\theta_5) & \sin(\theta_5)\cos(\theta_4) \\
-\sin(\theta_4)\cos(\theta_5)\cos(\theta_6) - \sin(\theta_6)\cos(\theta_4) & \sin(\theta_4)\sin(\theta_6)\cos(\theta_5) - \cos(\theta_4)\cos(\theta_6) & \sin(\theta_4)\sin(\theta_5) \\
\sin(\theta_5)\cos(\theta_6) & -\sin(\theta_5)\sin(\theta_6) & \cos(\theta_5)
\end{smallmatrix}
\]


In order to find $q_4$ we can solve $\frac{sin(\theta_4) \sin(\theta_5)}{\sin(\theta_5) \cos(\theta_4)}$. 

If $\sin(\theta_5) \neq 0$, then $q_4 = \arctan(\frac{\sin(\theta_4)}{\cos(\theta_4)})$. 


\subsection{$q_5$}

\[\left(\sin(\theta_5) \cos(\theta_4)\right)^2 + \left(\sin(\theta_4) \sin(\theta_5)\right)^2 = \sin(\theta_5)^2\]
\[ \theta_5= \arctan2(\sqrt{\left(\sin(\theta_5) \cos(\theta_4)\right)^2 + \left(\sin(\theta_4) \sin(\theta_5)\right)^2}, \cos(\theta_5)) \]

% sin5 = 0
% if np.sign(R36[1][2])/np.sign(np.sin(q4)) > 0:
%   sin5 = np.sqrt(R36[0][2]**2 + R36[1][2]**2)
% elif np.sign(R36[1][2])/np.sin(q4) < 0:
%   sin5 = -np.sqrt(R36[0][2]**2 + R36[1][2]**2)

% q5 = np.arctan2(sin5, R36[2][2])

\subsection{$q_6$}

Analogically to $q_4$, we can find the value of $q_6$ from ${}^3R_6$:

\[\frac{-\sin(\theta_5) \sin(\theta_6)}{\sin(\theta_5) \cos(\theta_6)}\]

If $\sin(\theta_5) \neq 0$, then $q_6 = \arctan(\frac{-\sin(\theta_6)}{\cos(\theta_6)})$. 




% Position $P_c$ equals ${}^0T_3$:

% \[{}^0T_3 = {}^0T_1 {}^1T_2 {}^2T_3\]

% \[{}^0T_3 = R_z(\theta_1^*) T_z(l_1) R_z(\frac{\pi}{2}) R_x(\frac{\pi}{2}) R_z(\theta_2^*) T_x(l_2) R_z(\theta_3^*) T_x(l_3) \]


% \[
% \begin{bmatrix}
%     \sin(\theta_1)\sin(\theta_2)\cos(\theta_3) + \sin(\theta_1)\sin(\theta_3)\cos(\theta_2) & \cos(\theta_1) & \sin(\theta_1)\sin(\theta_2)\sin(\theta_3) - \sin(\theta_1)\cos(\theta_2)\cos(\theta_3) & -l_2\sin(\theta_1)\cos(\theta_2) + l_3\sin(\theta_1)\sin(\theta_2)\sin(\theta_3) - l_3\sin(\theta_1)\cos(\theta_2)\cos(\theta_3) \\
%     -\sin(\theta_2)\cos(\theta_1)\cos(\theta_3) - \sin(\theta_3)\cos(\theta_1)\cos(\theta_2) & \sin(\theta_1) & -\sin(\theta_2)\sin(\theta_3)\cos(\theta_1) + \cos(\theta_1)\cos(\theta_2)\cos(\theta_3) & l_2\cos(\theta_1)\cos(\theta_2) - l_3\sin(\theta_2)\sin(\theta_3)\cos(\theta_1) + l_3\cos(\theta_1)\cos(\theta_2)\cos(\theta_3) \\
%     -\sin(\theta_2)\sin(\theta_3) + \cos(\theta_2)\cos(\theta_3) & 0 & \sin(\theta_2)\cos(\theta_3) + \sin(\theta_3)\cos(\theta_2) & l_1 + l_2\sin(\theta_2) + l_3\sin(\theta_2)\cos(\theta_3) + l_3\sin(\theta_3)\cos(\theta_2) \\
%     0 & 0 & 0 & 1
% \end{bmatrix}
% \]

% We can retrieve the coordinates of the wrist center from the rightmost column of ${}^0T_3$.

% \[
% \begin{cases}    
%     x = -l_2 \sin(\theta_1) \cos(\theta_2) + l_3 \sin(\theta_1) \sin(\theta_2) \sin(\theta_3) - l_3 \sin(\theta_1) \cos(\theta_2) \cos(\theta_3) \\
%     y = l_2 \cos(\theta_1) \cos(\theta_2) - l_3 \sin(\theta_2) \sin(\theta_3) \cos(\theta_1) + l_3 \cos(\theta_1) \cos(\theta_2) \cos(\theta_3) \\
%     z = l_1 + l_2 \sin(\theta_2) + l_3 \sin(\theta_2) \cos(\theta_2) + l_3 \sin(\theta_3) \cos(\theta_2)
% \end{cases}    
% \]

% \[
% \begin{cases}    
%     x =  -\sin(\theta_1)(l_2 \cos(\theta_2) - l_3 \sin(\theta_2) \sin(\theta_3) + l_3 \cos(\theta_2) \cos(\theta_3)) \\
%     y = \cos(\theta_1) (l_2  \cos(\theta_2) - l_3 \sin(\theta_2) \sin(\theta_3) + l_3 \cos(\theta_2) \cos(\theta_3)) \\
%     z = l_1 + l_2 \sin(\theta_2) + l_3 \sin(\theta_2) \cos(\theta_2) + l_3 \sin(\theta_3) \cos(\theta_2)
% \end{cases}    
% \]

% \[-\tan(\theta_1) = -\frac{\sin(\theta_1)}{\cos(\theta_1)} = \dfrac{x}{y}\]

% \[\theta_1 = -\arctan2(\dfrac{x}{y})\]


% Thus, 
% \begin{align*}
%     x_w &= \\
%     y_w &= \\
%     z_w &= 
% \end{align*}

% Since we have already calculated $q_1, q_2, q_3$, we can calculate $q_4, q_5, q_6$. 

% \[{}^3R_6 = ({}^0R_3^{-1}) {}^0R_6\]



% \[\]


\subsection{Singularity analysis}

Singularity occurs when:
\begin{enumerate}
    \item $\theta_3 = 0$ or $\theta_3 = \pi$. 
    \item $\theta_5 = 0$ or $\theta_5 = \pi$. 
\end{enumerate}

When a singularity situation arises in the inverse kinematics solution, 
we can revert back to the previous non-singular solution for this joint.

For specific examples of inverse kinematics solutions for different configurations, please refer to the code in the Colab link. 

\section{Task 4. Derive the jacobian matrix for your robot model.}

\[ J = [J_1 \quad J_2 \quad \ldots \quad J_6] \]

In screw theory, the Jacobian is defined as:

\[ J = \begin{bmatrix}
    U_{i-1} \times (O_n - O_{i-1}) \\
    U_{i-1}
\end{bmatrix} \]

    
$U_{i-1}$ is the column vector corresponding to the axis of rotation matrix (in our case, z). 
$O_{i-1}$ is the translation vector of transformation matrix for $T_{i-1}$. \\

\underline{Singularity}

To check the singularity of the Jacobian, we need to check whether its determinant is not 
equal to 0. 

Alternatively, we can check the rank of the matrix. A non-singular matrix is full-rank, whereas a singular matrix is not full-rank.

\section{Task 5-6.}
\textbf{Task 5. Plan a synchronized trajectory for all 6 joints between two poses. (consider 20Hz
controller frequency).\\}

\textbf{Task 6. Use the Jacobian matrix to check for singularities along the planned trajectory.}

\subsection{Trajectory Time}
Based on the $q$ parameters $[q_0, q_f, \dot{q}_{\text{max}}, \ddot{q}_{\text{max}}]$, we can determine the velocity profile and calculate the time for the trajectory.

\textbf{Triangular profile:}
\[
\text{If } \sqrt{\delta_q \times \ddot{q}_{\text{max}}} \leq \dot{q}_{\text{max}}, \text{ then:}
\]
\[
\tau = \frac{\delta_q}{\dot{q}_{\text{max}}}, \quad T = \tau, \quad tf = 2\tau
\]

\textbf{Trapezoidal profile:}
\[
\text{Otherwise, we have:}
\]
\[
T = \frac{\delta_q}{\dot{q}_{\text{max}}}, \quad \tau = \frac{\dot{q}_{\text{max}}}{\ddot{q}_{\text{max}}}
\]


Based on the time parameters obtained, we can determine the trajectory for the trapezoidal profile in the following way:
\[
\begin{cases}
q = q_0 + \frac{1}{2} \ddot{q}_{\text{max}} \cdot (t_i - t_0)^2, & \quad 0 < t_i \leq \tau \\
v = \min(\ddot{q}_{\text{max}} \cdot t, \dot{q}_{\text{max}}) \\
a = \ddot{q}_{\text{max}}, & \quad 0 < t_i \leq \tau
\end{cases}
\]


\[
\begin{cases}
q = q_0 + dq_0 \cdot (t_i - \tau), \quad \tau < t_i \leq T\\
v = \dot{q}_{\text{max}}\\
a = 0
\end{cases}
\]

\[
\begin{cases}
q = q_{\text{T}} + \dot{q}_{\text{T}} \cdot (t_i - T) - \frac{1}{2} \ddot{q}_{\text{max}} \cdot (t_i - T)^2, \quad T < t_i \leq t_f\\
v = \min(\ddot{q}_{\text{max}} \cdot (t_f - t_i), \dot{q}_{\text{max}})\\
a = -\ddot{q}_{\text{max}}
\end{cases}
\]

\subsection{Synchronization}
In the case of multiple joints, trajectory planning needs to be synchronized based on time.
To incorporate control with a frequency of 20 Hz, we can discretize time parameters.
Additionally, along the trajectory, we track the Jacobian to ensure a non-zero determinant.

For more details, please, refer to the code in colab link.

\end{document}