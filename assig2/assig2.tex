\documentclass{article}
\usepackage{graphicx} % Required for inserting images
\usepackage{hyperref}
\usepackage{caption} % For custom captions
\usepackage{amsmath}
\usepackage{multicol}
\usepackage{subcaption}


\title{Fundamentals of Robotics: Assignment 2}
\author{Ekaterina Mozhegova}
\date{April 20, 2024}

\begin{document}

\maketitle

\section{Link}

\href{https://colab.research.google.com/drive/170z2YVaeu9kHEuC4bXIWQqJ82eWDJoS2?usp=sharing}{Colab}

\section{Configuration of the chosen robot}

Manipulator with anntropomorphic elbow and a spherical wrist. 

\begin{center}
    \includegraphics*[width=0.85\textwidth]{images/new_manipulator.png}
\end{center}

Forward kinematics:
\[{}^0T_1 = R_z(\theta_1^*) T_z(l_1) R_z(\frac{\pi}{2}) R_x(\frac{\pi}{2})\]
\[{}^1T_2 = R_z(\theta_2^*) T_x(l_2) \]
\[{}^2T_3 = R_z(\theta_3^*) T_x(l_3) \]
\[{}^3T_4 = R_z(\theta_4^*) T_z(l_4) R_x(-\dfrac{\pi}{2}) R_z(-\pi)\]
\[{}^4T_5 = R_z(\theta_5^*) T_y(l_5) R_x(-\dfrac{\pi}{2})\]
\[{}^5T_6 = R_z(\theta_6^*)*T_z(l_6)\]

\[{}^0T_6 = {}^0T_1 {}^1T_2 {}^2T_3 {}^3T_4 {}^4T_5 {}^5T_6 \]

\section{Task 1. Derive inverse kinematics for your robot model}


Given:
End-effector position $O_6$ and orientation $R_6$.

Position $P_c$ equals $0T_3$:

\[{}^0T_3 = {}^0T_1 {}^1T_2 {}^2T_3\]

The translation column corresponds to the position of the wrist center. 

Thus, 
\begin{align*}
    x_w &= \\
    y_w &= \\
    z_w &= 
\end{align*}

Since we have already calculated $q_1, q_2, q_3$, we can calculate $q_4, q_5, q_6$. 

\[{}^3R_6 = {}^0R_3^{-1} {}^0R_6\]





\section{Task 2. Solve inverse kinematics for multiple positions}

\begin{enumerate}
    \item Solve the first 3 joints for positioning the wrist
    \item Solve the last 3 jooints for orienting the tool
\end{enumerate}

Target reference frame: 
\[T = R_T& t_T\]

\[\]


Target wrist point: $p_w = t_{T} - l_6 z_T$

Since $0R_6 = 0R_3 \cdot 3R_6$. 
\section{Task 3. Track the number of solutions along the way and choose the correct one and closest to
the previous (current) configuration.}
\section{Task 4. Derive the jacobian matrix for your robot model.}
\section{Task 5. Plan a synchronized trajectory for all 6 joints between two poses. (consider 20Hz
controller frequency)}
\section{Task 5. Use the Jacobian matrix to check for singularities along the planned trajectory}
\end{document}