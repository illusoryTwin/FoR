\documentclass{article}
\usepackage{graphicx} % Required for inserting images
\usepackage{hyperref}
\usepackage{caption} % For custom captions
\usepackage{amsmath}
\usepackage{multicol}
\usepackage{subcaption}
\usepackage{amsmath}

\title{Fundamentals of Robotics: Assignment 2}
\author{Ekaterina Mozhegova}
\date{April 20, 2024}

\begin{document}

\maketitle

\section{Link}

\href{https://colab.research.google.com/drive/1hU13PXIqbAtPv2cby5qNB10PobzW39JC?usp=sharing}{Colab}

\section{Configuration of the chosen robot}

Manipulator with anntropomorphic elbow and a spherical wrist. 

\begin{center}
    \includegraphics*[width=0.85\textwidth]{images/new_manipulator.png}
\end{center}

Forward kinematics:
\[{}^0T_1 = R_z(\theta_1^*) T_z(l_1) R_z(\frac{\pi}{2}) R_x(\frac{\pi}{2})\]
\[{}^1T_2 = R_z(\theta_2^*) T_x(l_2) \]
\[{}^2T_3 = R_z(\theta_3^*) T_x(l_3) \]
\[{}^3T_4 = R_z(\theta_4^*) T_z(l_4) R_x(-\dfrac{\pi}{2}) R_z(-\pi)\]
\[{}^4T_5 = R_z(\theta_5^*) T_y(l_5) R_x(-\dfrac{\pi}{2})\]
\[{}^5T_6 = R_z(\theta_6^*) T_z(l_6) \]

\[{}^0T_6 = {}^0T_1 {}^1T_2 {}^2T_3 {}^3T_4 {}^4T_5 {}^5T_6 \]

Let's define its geometry:
\[l_1 = 1, l_2 = 0.5, l_3 = 0.4, l_4 =0.4, l_5=0.5, l_6 = 0.4\]

And let's define the end-effector position:
\[x_6=1.5, y_6=0.5, z_6 = 1.5\]


\section{Task 1. Derive inverse kinematics for your robot model}


Given:
End-effector position $O_6$ and orientation $R_6$.

Position $P_c$ equals ${}^0T_3$:

\[{}^0T_3 = {}^0T_1 {}^1T_2 {}^2T_3\]

\[{}^0T_3 = R_z(\theta_1^*) T_z(l_1) R_z(\frac{\pi}{2}) R_x(\frac{\pi}{2}) R_z(\theta_2^*) T_x(l_2) R_z(\theta_3^*) T_x(l_3) \]


\[
\begin{bmatrix}
    \sin(\theta_1)\sin(\theta_2)\cos(\theta_3) + \sin(\theta_1)\sin(\theta_3)\cos(\theta_2) & \cos(\theta_1) & \sin(\theta_1)\sin(\theta_2)\sin(\theta_3) - \sin(\theta_1)\cos(\theta_2)\cos(\theta_3) & -l_2\sin(\theta_1)\cos(\theta_2) + l_3\sin(\theta_1)\sin(\theta_2)\sin(\theta_3) - l_3\sin(\theta_1)\cos(\theta_2)\cos(\theta_3) \\
    -\sin(\theta_2)\cos(\theta_1)\cos(\theta_3) - \sin(\theta_3)\cos(\theta_1)\cos(\theta_2) & \sin(\theta_1) & -\sin(\theta_2)\sin(\theta_3)\cos(\theta_1) + \cos(\theta_1)\cos(\theta_2)\cos(\theta_3) & l_2\cos(\theta_1)\cos(\theta_2) - l_3\sin(\theta_2)\sin(\theta_3)\cos(\theta_1) + l_3\cos(\theta_1)\cos(\theta_2)\cos(\theta_3) \\
    -\sin(\theta_2)\sin(\theta_3) + \cos(\theta_2)\cos(\theta_3) & 0 & \sin(\theta_2)\cos(\theta_3) + \sin(\theta_3)\cos(\theta_2) & l_1 + l_2\sin(\theta_2) + l_3\sin(\theta_2)\cos(\theta_3) + l_3\sin(\theta_3)\cos(\theta_2) \\
    0 & 0 & 0 & 1
\end{bmatrix}
\]

We can retrieve the coordinates of the wrist center from the rightmost column of ${}^0T_3$.

\[
\begin{cases}    
    x = -l_2 \sin(\theta_1) \cos(\theta_2) + l_3 \sin(\theta_1) \sin(\theta_2) \sin(\theta_3) - l_3 \sin(\theta_1) \cos(\theta_2) \cos(\theta_3) \\
    y = l_2 \cos(\theta_1) \cos(\theta_2) - l_3 \sin(\theta_2) \sin(\theta_3) \cos(\theta_1) + l_3 \cos(\theta_1) \cos(\theta_2) \cos(\theta_3) \\
    z = l_1 + l_2 \sin(\theta_2) + l_3 \sin(\theta_2) \cos(\theta_2) + l_3 \sin(\theta_3) \cos(\theta_2)
\end{cases}    
\]

\[
\begin{cases}    
    x =  -\sin(\theta_1)(l_2 \cos(\theta_2) - l_3 \sin(\theta_2) \sin(\theta_3) + l_3 \cos(\theta_2) \cos(\theta_3)) \\
    y = \cos(\theta_1) (l_2  \cos(\theta_2) - l_3 \sin(\theta_2) \sin(\theta_3) + l_3 \cos(\theta_2) \cos(\theta_3)) \\
    z = l_1 + l_2 \sin(\theta_2) + l_3 \sin(\theta_2) \cos(\theta_2) + l_3 \sin(\theta_3) \cos(\theta_2)
\end{cases}    
\]

\[-\tan(\theta_1) = -\frac{\sin(\theta_1)}{\cos(\theta_1)} = \dfrac{x}{y}\]

\[\theta_1 = -\arctan2(\dfrac{x}{y})\]


Thus, 
\begin{align*}
    x_w &= \\
    y_w &= \\
    z_w &= 
\end{align*}

Since we have already calculated $q_1, q_2, q_3$, we can calculate $q_4, q_5, q_6$. 

\[{}^3R_6 = ({}^0R_3^{-1}) {}^0R_6\]



\[\]



\section{Task 2. Solve inverse kinematics for multiple positions}

\begin{enumerate}
    \item Solve the first 3 joints for positioning the wrist
    \item Solve the last 3 jooints for orienting the tool
\end{enumerate}

Target reference frame: 
% \[T = R_T& t_T\]

% \[\]


Target wrist point: $p_w = t_{T} - l_6 z_T$

Since $0R_6 = 0R_3 \cdot 3R_6$. 
\section{Task 3. Track the number of solutions along the way and choose the correct one and closest to
the previous (current) configuration.}
\section{Task 4. Derive the jacobian matrix for your robot model.}


\section{Task 5. Plan a synchronized trajectory for all 6 joints between two poses. (consider 20Hz
controller frequency)}
\section{Task 5. Use the Jacobian matrix to check for singularities along the planned trajectory}
\end{document}